%%*************************************************************************
%% Legal Notice:
%% This code is offered as-is without any warranty either expressed or
%% implied; without even the implied warranty of MERCHANTABILITY or
%% FITNESS FOR A PARTICULAR PURPOSE! 
%% User assumes all risk.
%% In no event shall the IEEE or any contributor to this code be liable for
%% any damages or losses, including, but not limited to, incidental,
%% consequential, or any other damages, resulting from the use or misuse
%% of any information contained here.
%%
%% All comments are the opinions of their respective authors and are not
%% necessarily endorsed by the IEEE.
%%
%% This work is distributed under the LaTeX Project Public License (LPPL)
%% ( http://www.latex-project.org/ ) version 1.3, and may be freely used,
%% distributed and modified. A copy of the LPPL, version 1.3, is included
%% in the base LaTeX documentation of all distributions of LaTeX released
%% 2003/12/01 or later.
%% Retain all contribution notices and credits.
%% ** Modified files should be clearly indicated as such, including  **
%% ** renaming them and changing author support contact information. **
%%*************************************************************************


% *** Authors should verify (and, if needed, correct) their LaTeX system  ***
% *** with the testflow diagnostic prior to trusting their LaTeX platform ***
% *** with production work. The IEEE's font choices and paper sizes can   ***
% *** trigger bugs that do not appear when using other class files.       ***                          ***
% The testflow support page is at:
% http://www.michaelshell.org/tex/testflow/



\documentclass[conference]{IEEEtran}
% Some Computer Society conferences also require the compsoc mode option,
% but others use the standard conference format.
%
% If IEEEtran.cls has not been installed into the LaTeX system files,
% manually specify the path to it like:
% \documentclass[conference]{../sty/IEEEtran}





% Some very useful LaTeX packages include:
% (uncomment the ones you want to load)


% *** MISC UTILITY PACKAGES ***
%
%\usepackage{ifpdf}
% Heiko Oberdiek's ifpdf.sty is very useful if you need conditional
% compilation based on whether the output is pdf or dvi.
% usage:
% \ifpdf
%   % pdf code
% \else
%   % dvi code
% \fi
% The latest version of ifpdf.sty can be obtained from:
% http://www.ctan.org/pkg/ifpdf
% Also, note that IEEEtran.cls V1.7 and later provides a builtin
% \ifCLASSINFOpdf conditional that works the same way.
% When switching from latex to pdflatex and vice-versa, the compiler may
% have to be run twice to clear warning/error messages.






% *** CITATION PACKAGES ***
%
\usepackage{cite}
% cite.sty was written by Donald Arseneau
% V1.6 and later of IEEEtran pre-defines the format of the cite.sty package
% \cite{} output to follow that of the IEEE. Loading the cite package will
% result in citation numbers being automatically sorted and properly
% "compressed/ranged". e.g., [1], [9], [2], [7], [5], [6] without using
% cite.sty will become [1], [2], [5]--[7], [9] using cite.sty. cite.sty's
% \cite will automatically add leading space, if needed. Use cite.sty's
% noadjust option (cite.sty V3.8 and later) if you want to turn this off
% such as if a citation ever needs to be enclosed in parenthesis.
% cite.sty is already installed on most LaTeX systems. Be sure and use
% version 5.0 (2009-03-20) and later if using hyperref.sty.
% The latest version can be obtained at:
% http://www.ctan.org/pkg/cite
% The documentation is contained in the cite.sty file itself.



% *** GRAPHICS RELATED PACKAGES ***
%
\ifCLASSINFOpdf
  % \usepackage[pdftex]{graphicx}
  % declare the path(s) where your graphic files are
  % \graphicspath{{../pdf/}{../jpeg/}}
  % and their extensions so you won't have to specify these with
  % every instance of \includegraphics
  % \DeclareGraphicsExtensions{.pdf,.jpeg,.png}
\else
  % or other class option (dvipsone, dvipdf, if not using dvips). graphicx
  % will default to the driver specified in the system graphics.cfg if no
  % driver is specified.
  % \usepackage[dvips]{graphicx}
  % declare the path(s) where your graphic files are
  % \graphicspath{{../eps/}}
  % and their extensions so you won't have to specify these with
  % every instance of \includegraphics
  % \DeclareGraphicsExtensions{.eps}
\fi
% graphicx was written by David Carlisle and Sebastian Rahtz. It is
% required if you want graphics, photos, etc. graphicx.sty is already
% installed on most LaTeX systems. The latest version and documentation
% can be obtained at: 
% http://www.ctan.org/pkg/graphicx
% Another good source of documentation is "Using Imported Graphics in
% LaTeX2e" by Keith Reckdahl which can be found at:
% http://www.ctan.org/pkg/epslatex
%
% latex, and pdflatex in dvi mode, support graphics in encapsulated
% postscript (.eps) format. pdflatex in pdf mode supports graphics
% in .pdf, .jpeg, .png and .mps (metapost) formats. Users should ensure
% that all non-photo figures use a vector format (.eps, .pdf, .mps) and
% not a bitmapped formats (.jpeg, .png). The IEEE frowns on bitmapped formats
% which can result in "jaggedy"/blurry rendering of lines and letters as
% well as large increases in file sizes.
%
% You can find documentation about the pdfTeX application at:
% http://www.tug.org/applications/pdftex





% *** MATH PACKAGES ***
%
%\usepackage{amsmath}
% A popular package from the American Mathematical Society that provides
% many useful and powerful commands for dealing with mathematics.
%
% Note that the amsmath package sets \interdisplaylinepenalty to 10000
% thus preventing page breaks from occurring within multiline equations. Use:
%\interdisplaylinepenalty=2500
% after loading amsmath to restore such page breaks as IEEEtran.cls normally
% does. amsmath.sty is already installed on most LaTeX systems. The latest
% version and documentation can be obtained at:
% http://www.ctan.org/pkg/amsmath





% *** SPECIALIZED LIST PACKAGES ***
%
%\usepackage{algorithmic}
% algorithmic.sty was written by Peter Williams and Rogerio Brito.
% This package provides an algorithmic environment fo describing algorithms.
% You can use the algorithmic environment in-text or within a figure
% environment to provide for a floating algorithm. Do NOT use the algorithm
% floating environment provided by algorithm.sty (by the same authors) or
% algorithm2e.sty (by Christophe Fiorio) as the IEEE does not use dedicated
% algorithm float types and packages that provide these will not provide
% correct IEEE style captions. The latest version and documentation of
% algorithmic.sty can be obtained at:
% http://www.ctan.org/pkg/algorithms
% Also of interest may be the (relatively newer and more customizable)
% algorithmicx.sty package by Szasz Janos:
% http://www.ctan.org/pkg/algorithmicx




% *** ALIGNMENT PACKAGES ***
%
%\usepackage{array}
% Frank Mittelbach's and David Carlisle's array.sty patches and improves
% the standard LaTeX2e array and tabular environments to provide better
% appearance and additional user controls. As the default LaTeX2e table
% generation code is lacking to the point of almost being broken with
% respect to the quality of the end results, all users are strongly
% advised to use an enhanced (at the very least that provided by array.sty)
% set of table tools. array.sty is already installed on most systems. The
% latest version and documentation can be obtained at:
% http://www.ctan.org/pkg/array


% IEEEtran contains the IEEEeqnarray family of commands that can be used to
% generate multiline equations as well as matrices, tables, etc., of high
% quality.




% *** SUBFIGURE PACKAGES ***
%\ifCLASSOPTIONcompsoc
%  \usepackage[caption=false,font=normalsize,labelfont=sf,textfont=sf]{subfig}
%\else
%  \usepackage[caption=false,font=footnotesize]{subfig}
%\fi
% subfig.sty, written by Steven Douglas Cochran, is the modern replacement
% for subfigure.sty, the latter of which is no longer maintained and is
% incompatible with some LaTeX packages including fixltx2e. However,
% subfig.sty requires and automatically loads Axel Sommerfeldt's caption.sty
% which will override IEEEtran.cls' handling of captions and this will result
% in non-IEEE style figure/table captions. To prevent this problem, be sure
% and invoke subfig.sty's "caption=false" package option (available since
% subfig.sty version 1.3, 2005/06/28) as this is will preserve IEEEtran.cls
% handling of captions.
% Note that the Computer Society format requires a larger sans serif font
% than the serif footnote size font used in traditional IEEE formatting
% and thus the need to invoke different subfig.sty package options depending
% on whether compsoc mode has been enabled.
%
% The latest version and documentation of subfig.sty can be obtained at:
% http://www.ctan.org/pkg/subfig




% *** FLOAT PACKAGES ***
%
%\usepackage{fixltx2e}
% fixltx2e, the successor to the earlier fix2col.sty, was written by
% Frank Mittelbach and David Carlisle. This package corrects a few problems
% in the LaTeX2e kernel, the most notable of which is that in current
% LaTeX2e releases, the ordering of single and double column floats is not
% guaranteed to be preserved. Thus, an unpatched LaTeX2e can allow a
% single column figure to be placed prior to an earlier double column
% figure.
% Be aware that LaTeX2e kernels dated 2015 and later have fixltx2e.sty's
% corrections already built into the system in which case a warning will
% be issued if an attempt is made to load fixltx2e.sty as it is no longer
% needed.
% The latest version and documentation can be found at:
% http://www.ctan.org/pkg/fixltx2e


%\usepackage{stfloats}
% stfloats.sty was written by Sigitas Tolusis. This package gives LaTeX2e
% the ability to do double column floats at the bottom of the page as well
% as the top. (e.g., "\begin{figure*}[!b]" is not normally possible in
% LaTeX2e). It also provides a command:
%\fnbelowfloat
% to enable the placement of footnotes below bottom floats (the standard
% LaTeX2e kernel puts them above bottom floats). This is an invasive package
% which rewrites many portions of the LaTeX2e float routines. It may not work
% with other packages that modify the LaTeX2e float routines. The latest
% version and documentation can be obtained at:
% http://www.ctan.org/pkg/stfloats
% Do not use the stfloats baselinefloat ability as the IEEE does not allow
% \baselineskip to stretch. Authors submitting work to the IEEE should note
% that the IEEE rarely uses double column equations and that authors should try
% to avoid such use. Do not be tempted to use the cuted.sty or midfloat.sty
% packages (also by Sigitas Tolusis) as the IEEE does not format its papers in
% such ways.
% Do not attempt to use stfloats with fixltx2e as they are incompatible.
% Instead, use Morten Hogholm'a dblfloatfix which combines the features
% of both fixltx2e and stfloats:
%
% \usepackage{dblfloatfix}
% The latest version can be found at:
% http://www.ctan.org/pkg/dblfloatfix




% *** PDF, URL AND HYPERLINK PACKAGES ***
%
%\usepackage{url}
% url.sty was written by Donald Arseneau. It provides better support for
% handling and breaking URLs. url.sty is already installed on most LaTeX
% systems. The latest version and documentation can be obtained at:
% http://www.ctan.org/pkg/url
% Basically, \url{my_url_here}.




% *** Do not adjust lengths that control margins, column widths, etc. ***
% *** Do not use packages that alter fonts (such as pslatex).         ***
% There should be no need to do such things with IEEEtran.cls V1.6 and later.
% (Unless specifically asked to do so by the journal or conference you plan
% to submit to, of course. )


% correct bad hyphenation here
\hyphenation{op-tical net-works semi-conduc-tor}


\begin{document}
%
% paper title
% Titles are generally capitalized except for words such as a, an, and, as,
% at, but, by, for, in, nor, of, on, or, the, to and up, which are usually
% not capitalized unless they are the first or last word of the title.
% Linebreaks \\ can be used within to get better formatting as desired.
% Do not put math or special symbols in the title.
\title{WSN for Livestock Monitoring -- Related Work and Design}

% author names and affiliations
% use a multiple column layout for up to three different
% affiliations
\author{\IEEEauthorblockN{
Herman Lundkvist\IEEEauthorrefmark{1},
Attila Para\IEEEauthorrefmark{2},
Vipul Mahawar\IEEEauthorrefmark{3} and
Omar Elshal\IEEEauthorrefmark{4}}
\IEEEauthorblockA{\IEEEauthorrefmark{1}Student Id: 0973534\\
Email: h.e.lundkvist@student.tue.nl}
\IEEEauthorblockA{\IEEEauthorrefmark{2}Student Id: 0975194\\
Email: a.para@student.tue.nl}
\IEEEauthorblockA{\IEEEauthorrefmark{3}Student Id: 09xxxxx\\
Email: v.mahawar@student.tue.nl}
\IEEEauthorblockA{\IEEEauthorrefmark{4}Student Id: 0980295\\
Email: o.a.m.elshal@student.tue.nl}}

% conference papers do not typically use \thanks and this command
% is locked out in conference mode. If really needed, such as for
% the acknowledgment of grants, issue a \IEEEoverridecommandlockouts
% after \documentclass

% for over three affiliations, or if they all won't fit within the width
% of the page, use this alternative format:
% 
%\author{\IEEEauthorblockN{Michael Shell\IEEEauthorrefmark{1},
%Homer Simpson\IEEEauthorrefmark{2},
%James Kirk\IEEEauthorrefmark{3}, 
%Montgomery Scott\IEEEauthorrefmark{3} and
%Eldon Tyrell\IEEEauthorrefmark{4}}
%\IEEEauthorblockA{\IEEEauthorrefmark{1}School of Electrical and Computer Engineering\\
%Georgia Institute of Technology,
%Atlanta, Georgia 30332--0250\\ Email: see http://www.michaelshell.org/contact.html}
%\IEEEauthorblockA{\IEEEauthorrefmark{2}Twentieth Century Fox, Springfield, USA\\
%Email: homer@thesimpsons.com}
%\IEEEauthorblockA{\IEEEauthorrefmark{3}Starfleet Academy, San Francisco, California 96678-2391\\
%Telephone: (800) 555--1212, Fax: (888) 555--1212}
%\IEEEauthorblockA{\IEEEauthorrefmark{4}Tyrell Inc., 123 Replicant Street, Los Angeles, California 90210--4321}}




% use for special paper notices
%\IEEEspecialpapernotice{(Invited Paper)}




% make the title area
\maketitle

% For peer review papers, you can put extra information on the cover
% page as needed:
% \ifCLASSOPTIONpeerreview
% \begin{center} \bfseries EDICS Category: 3-BBND \end{center}
% \fi
%
% For peerreview papers, this IEEEtran command inserts a page break and
% creates the second title. It will be ignored for other modes.
\IEEEpeerreviewmaketitle


\section{Introduction}

\subsection{Background}

The livestock industry constitutes a considerable part of the world’s economy.
In fact, it generates some € 8.6 billion every year in the Netherlands alone.
~\cite{ned_gov} At the same time, there is a clear trend of automatization in
this sector that aims to increase the efficiency and decrease the amount of
human labour.

However, one of the major costs for the farmers in this field, is diseases
contracted by their animals. By developing a system that could detect such
diseases, and other abnormal behavior, one could potentially reduce this costs
by a great amount.  


\subsection{Application Description}

The main purpose of this project is to design a system that can help farmers
monitor the health of cattle while they are grazing. This will be done using
wireless sensor network (WSN) technology, because of its ability to deliver
real-time monitoring at a very low cost.  However, on account of range
limitations of WSN transceivers, the system will be designed for a relatively
small field of 10 ha which is in the range of an average dairy farm in the
Netherlands ~\cite{agricultural_systems}.

The network of the system will consist of three types of nodes: a base station,
which acts as a data sink; sensor nodes, one for each head of cattle, recording
health characteristics; and a number of relay nodes on fixed positions in the
field, forwarding data from the sensor nodes to the base station. The reason
for using relay nodes, is to be able to cover the majority of the field.

The system will be used both to detect different types of diseases: fever,
lameness and mastitis, and to detect if a cow is in estrus.  This can be
accomplished by the use accelerometer-, microphone-, and temperature sensors in
the sensor nodes.~\cite{hese2014}  Every 4 minutes, the sensor nodes will send the acquired
data in a processed form to the base station for storage. The base station can
analyse the sample values to detect patterns that correspond to different
diseases or the onset of estrus.

\subsection{Related Work}


\section{Application Specific Challenges}

\subsection{Mobility}

Mobility presents a major challenge for sensor nodes mounted on cattle since
they are subject to frequent changes in location.  The animal monitoring system
must be able to support animal mobility; wireless sensors are used to monitor
the health condition of animals moving freely around open fields. The network
topology and routing paths should therefore be dynamic, able to respond to
frequent animal movement while optimising packet delivery.

\subsection{Size of Field}

The key challenge in any wireless network is the coverage of the all the
wireless nodes. In the Cattle Monitoring WSN the network architecture should
account to cover the average size of fields and must be scalable when needed
accordingly for larger fields. As per now the average size of fields in
Netherlands is 100,000 square meters and maximum coverage done in Zigbee WSN
protocol is approx 500m so by using one or two relay nodes the sufficient area
of average cattle field can be covered.

\subsection{Wireless Communication}

As the animals move freely in the cattle.Wireless technology is considered the
only feasible method to establish and maintain communications between a base
station and network nodes attached to cattle animals. The wireless
communication has various challenges like signal attenuation in the medium,
interference from other radio signals, crosstalk from the other nodes, etc. As
the radio signal generated from the node is weak owing to preserve battery on
the node, a significant amount of wireless signal is attenuated from the animal
tissue. For maximum signal coverage from relay node the antenna on the animal
is placed on a neck belt with two antenna on both sides of the neck to have
spatial diversity. Also the radio element is switched off as soon as the data
transmission is finished so as to maximize battery life.

\subsection{Energy Consumption}

The battery usage is one the major constraint in the Cattle Monitoring System
because the radio collar used for cattle monitoring is expected to run for
5 years without battery replacement. As there is limited battery power
available per node the design should account for low powered, lightweight radio
antenna. The network protocol should be designed so as to use limited battery
power and only communicate with the base station, to provide adequate amount of
data required for monitoring health of cattle. The processing capabilities of
the node should be in a way that it consumes minimum amount of power from the
battery and perform the necessary operations on the data as requested by the
network protocol.

\subsection{Cost of the System}

The Wireless Sensor Network developed will be used to monitor the health of the
cattle which in turn should reduce the cost of keeping the cattle in good
health. For a cattle WSN it should be cheap and the wireless nodes must be
low-cost with high lifespan and low maintenance to reduce the average cost of
maintaining the cattle. Another reason for sensor nodes to be low-cost is the
requirement of potentially high number of nodes needed for monitoring an entire
herd of cattle. Also as the WSN becomes more widespread and general the average
cost of having a WSN monitoring system will reduce considerably over a period
of time.  


\section{Scope}


\section{Requirements}

To ensure that the designed system can meet the needs of real-world usage, it
needs to meet the following requirements:

\begin{enumerate} 

    \item The network must be implemented in such a way that If a sensor node
        moves out of coverage of the network, the sensor node will be
        reassigned to the network once it moves into coverage again.  
        
    \item The end-to-end latency between the sensor nodes and the base station
        must be less than 1 minute for at least 90 \% of data packets.
        
    \item For every sensor node at least 70 \% of the recorded sensor data must
        be delivered to the base station every 12 hours.
        
    \item The time until the first sensor node of the network fails must be
        greater than one year.  
        
    \item The network must function and be scalable for up to 100 sensor nodes.
        
    \item The network must continue to function whether nodes are added or
        removed.
    
    \item When out of coverage, each sensor node must be able to store sensor
        values for a period of up to four hours.
    
    \item  sensor node must not be out of coverage for more than four hours at
        a time.

\end{enumerate}


\subsection{Data size}

One of the most important requirements for the system is to be able to handle
the amount of data generated by the sensors. Estimating this amount is thus
also very important. For storing the temperature of the cow after processing
the raw data from the sensors, one byte is enough to give a reasonable
resolution. This is because a dairy cow’s normal rectal temperature lies
between 38.3 and 38.9 centigrades ~\cite{wiki_bov} and one could limiting the
byte to represent values between 30 and 50 degrees centigrades without losing
any information.  Using the same argumentation, one byte could also be used to
store the average heart rate, because the normal heart rate for a dairy cow
lies between 40 and 84 beats per minute ~\cite{wiki_bov}.

In addition, to store the data gathered from the accelerometers, one could
devise a similar strategy to the one used in a study where the behaviour of
sows were classified.\cite{marchioro_sows_2011} In this study, the activities
of the animals were organized into four different sets, using a method
specifically developed for low powered embedded devices. This resulted in
a classification with an accuracy of close to 90 \%. Using data in this format,
the researchers were able to detect the onset of farrowing. In this study only
four sets of behaviours were used, if it is possible to develop similar sets
for behavioural studies on cows, one could potentially store this data using
only two bits per sample.

Using two samples for both the temperature and the heart rate, while also
rounding up to the nearest byte, total amount of data needed to be sent each
five minute interval would be five bytes.


\section{Protocol Stack}

The system will developed for platforms supporting ContikiOs. For the
physical- and MAC layers, the standard protocol IEEE 802.15.4
\cite{ieee_computer_society_ieee_2011} will be used.  For the network- and
application layer, features from ContikiOs will be used.  Moreover, a routing
algorithm might possibly be designed, if the routing algorithms present in
ContikiOs are deemed unfitting for our application.

\subsection{Physical Layer}

To mitigate the problems with coverage, tranceivers with carrier frequencies
between 902 and 928 of the IEEE 802.15.4 standard, will be considered in order
to get an increased range. Even though these frequencies provide a lower
datarate, of around 40 kbp/s \cite{ieee_computer_society_ieee_2011}, it is
still sufficient for the small amounts of data collected by the sensors.
Furthermore, the modulation scheme to be used will either be ASK-, BPSK or
O-QPSK.

\subsection{MAC Layer}

For the MAC layer, there are two modes to be used in the IEEE 802.15.4
standard: nonbeacon-enabled mode and beacon-enabled mode. In the former
a unslotted CSMA/CA mechanism is used. In the latter one, a more elaborate
scheme is used, wherein special nodes of the network, called coordinators,
regularly send out beacons to synchronize the nodes it is associated with. The
period between two beacons is called a superframe, and within this the nodes
use a slotted CSMA/CA mechanism.

Because of the need for synchronization, the beacon-enabled mode might cause
problems if a sensor node moves out of coverage of a coordinator for a longer
period of time, or if a sensor node needs to switch to a different coordinator.
The sensor node might have to listen for a very long time to receive the next
beacon, possibly even the whole time while it is out of range, thus consuming
a lot of power.

On the other hand, the unslotted CSMA/CA mechanism of the nonbeacon-enabled
mode, in addition to being less complex, does not require the nodes associated
with a coordinator to regularly receive a beacon, and is thus suitable for the
mobile nature of our application. The only drawback is the power consumption
required for continuously listening at regular intervals.  However, since the
relay nodes and the base station are the only ones that need to receive data,
only these have to listen, and not the sensor nodes. In view of these
arguments, the design will use the nonbeacon-enabled mode.



\section{Network Architecture}


\section{Simulation Software}

Compatible Hardware: TI MSP430x,  TI MSP430x, Atmel AVR, micaz, TI MSP430, TI
MSP430 and Native

Max Number of Nodes: The number of emulated nodes depends on memory and the
application running on nodes which can also affect simulation speed. As there
is a main simulation thread, using more than two cores (one for the simulation
and one for the rest) gives no significant improvement(9).

Transmission power: Max Power of tmote for example is 31. Transmission power
level can be controlled according to your needs using the function call
cc2420_set_txpower(desired_power_level), where desired_power_level could be any
of the following values (in the increasing order of intensity): 3, 7, 11, 15,
19, 23, 27, 31.

Mobility models Capability: The mobility plugin have to be installed in order
to be able to use mobility in Cooja. However, to be able to generate our own
specific mobility file, another tool like BonnMotion have to be used to make
the mobility output format compatible with Cooja’s mobility plugin format. To
simulate the movement of the cows, we will only be considering a few simple
mobility models like: gauss-markov, random waypoint and reference point group
mobility model. As simulating a more complex mobility patterns of the cows is
not the goal of this project. ~\cite{contikt_dev}

Simulated Environment Dimensions: To provide a reliable transmission range,
there is UDGM (Unit Disk Graph Radio Medium) implementation that can be
configured to change some parameters like transmission range and interference
range, but also depending on the transmission power parameter, the resulting
transmission range will differ.~\cite{contikt_dev}

Support for different Node Types (Relay, data sink, sensors) Cooja supports all
the different node types like sensors, relays and actuators.  Simply according
to your definition of each node.

Support for our MAC-layer/ protocol stack
Contiki supports different MAC layers: X-MAC, LLP (Low Power Probing), Simple TDMA and nullmac (non-persistent CSMA). X-MAC and LPP are low power MACs that work best under low traffic loads(10).
Furthermore, Contiki supports different protocol stacks like uIP which is a very small and fully compliant TCP/IP stack. Contiki supports also Rime stack which consists of small layers built on top of each other and has single and multi-hop broadcast(11).


\section{Work Plan}

% An example of a floating figure using the graphicx package.
% Note that \label must occur AFTER (or within) \caption.
% For figures, \caption should occur after the \includegraphics.
% Note that IEEEtran v1.7 and later has special internal code that
% is designed to preserve the operation of \label within \caption
% even when the captionsoff option is in effect. However, because
% of issues like this, it may be the safest practice to put all your
% \label just after \caption rather than within \caption{}.
%
% Reminder: the "draftcls" or "draftclsnofoot", not "draft", class
% option should be used if it is desired that the figures are to be
% displayed while in draft mode.
%
%\begin{figure}[!t]
%\centering
%\includegraphics[width=2.5in]{myfigure}
% where an .eps filename suffix will be assumed under latex, 
% and a .pdf suffix will be assumed for pdflatex; or what has been declared
% via \DeclareGraphicsExtensions.
%\caption{Simulation results for the network.}
%\label{fig_sim}
%\end{figure}

% Note that the IEEE typically puts floats only at the top, even when this
% results in a large percentage of a column being occupied by floats.


% An example of a double column floating figure using two subfigures.
% (The subfig.sty package must be loaded for this to work.)
% The subfigure \label commands are set within each subfloat command,
% and the \label for the overall figure must come after \caption.
% \hfil is used as a separator to get equal spacing.
% Watch out that the combined width of all the subfigures on a 
% line do not exceed the text width or a line break will occur.
%
%\begin{figure*}[!t]
%\centering
%\subfloat[Case I]{\includegraphics[width=2.5in]{box}%
%\label{fig_first_case}}
%\hfil
%\subfloat[Case II]{\includegraphics[width=2.5in]{box}%
%\label{fig_second_case}}
%\caption{Simulation results for the network.}
%\label{fig_sim}
%\end{figure*}
%
% Note that often IEEE papers with subfigures do not employ subfigure
% captions (using the optional argument to \subfloat[]), but instead will
% reference/describe all of them (a), (b), etc., within the main caption.
% Be aware that for subfig.sty to generate the (a), (b), etc., subfigure
% labels, the optional argument to \subfloat must be present. If a
% subcaption is not desired, just leave its contents blank,
% e.g., \subfloat[].


% An example of a floating table. Note that, for IEEE style tables, the
% \caption command should come BEFORE the table and, given that table
% captions serve much like titles, are usually capitalized except for words
% such as a, an, and, as, at, but, by, for, in, nor, of, on, or, the, to
% and up, which are usually not capitalized unless they are the first or
% last word of the caption. Table text will default to \footnotesize as
% the IEEE normally uses this smaller font for tables.
% The \label must come after \caption as always.
%
%\begin{table}[!t]
%% increase table row spacing, adjust to taste
%\renewcommand{\arraystretch}{1.3}
% if using array.sty, it might be a good idea to tweak the value of
% \extrarowheight as needed to properly center the text within the cells
%\caption{An Example of a Table}
%\label{table_example}
%\centering
%% Some packages, such as MDW tools, offer better commands for making tables
%% than the plain LaTeX2e tabular which is used here.
%\begin{tabular}{|c||c|}
%\hline
%One & Two\\
%\hline
%Three & Four\\
%\hline
%\end{tabular}
%\end{table}


% Note that the IEEE does not put floats in the very first column
% - or typically anywhere on the first page for that matter. Also,
% in-text middle ("here") positioning is typically not used, but it
% is allowed and encouraged for Computer Society conferences (but
% not Computer Society journals). Most IEEE journals/conferences use
% top floats exclusively. 
% Note that, LaTeX2e, unlike IEEE journals/conferences, places
% footnotes above bottom floats. This can be corrected via the
% \fnbelowfloat command of the stfloats package.


% conference papers do not normally have an appendix

% trigger a \newpage just before the given reference
% number - used to balance the columns on the last page
% adjust value as needed - may need to be readjusted if
% the document is modified later
%\IEEEtriggeratref{8}
% The "triggered" command can be changed if desired:
%\IEEEtriggercmd{\enlargethispage{-5in}}

% references section

% can use a bibliography generated by BibTeX as a .bbl file
% BibTeX documentation can be easily obtained at:
% http://mirror.ctan.org/biblio/bibtex/contrib/doc/
% The IEEEtran BibTeX style support page is at:
% http://www.michaelshell.org/tex/ieeetran/bibtex/
%\bibliographystyle{IEEEtran}
% argument is your BibTeX string definitions and bibliography database(s)
%\bibliography{IEEEabrv,../bib/paper}
%
% <OR> manually copy in the resultant .bbl file
% set second argument of \begin to the number of references
% (used to reserve space for the reference number labels box)
\begin{thebibliography}{1}

\bibitem{ned_gov}
https://www.government.nl/topics/agriculture-and-livestock/contents/livestock-farming, taken 12:30 22 sep 2015.

\bibitem{agricultural_systems}
D. B. Grigg, \emph{The Agricultural Systems of
the World: An Evolutionary Approach}, Cambridge University Press, november
1974, p. 188.

\bibitem{hese2014}
A. ~Helwatkar, D. ~Riordan and J. ~Walsh,
\emph{Sensor Technology For Animal Health Monitoring},
Proceedings of the 8th International Conference on Sensing Technology,
sep, 2014.

\bibitem{wiki_bov} 
https://en.wikivet.net/Bovine\_Physiology\_-\_WikiNormals, taken 12:30 22 sep
2015 

\bibitem{marchioro_sows_2011}
G. F. ~Marchioro, C. ~Cornou, A. R. ~Kristensen and J. ~Madsen,
\emph{Sows’ activity classification device using acceleration data – A resource constrained approach},
Computers and Electronics in Agriculture,
issn: 01681699,
jun,
2011,
pages 110--117,

\bibitem{ieee_computer_society_ieee_2011}
IEEE Computer Society, LAN/MAN Standards Committee, Institute of Electrical
and Electronics Engineers and IEEE-SA Standards Board, \emph{IEEE standard
for local and metropolitan area networks. Part 15.4}, isbn
978-0-7381-6684-1, Institute of Electrical and Electronics Engineers, 2011,

\bibitem{contikti_dev}
    
http://contiki-developers.narkive.com, taken 13.00 22 sep 2015.

%citation format
    %H.~Kopka and P.~W. Daly, \emph{A Guide to \LaTeX}, 3rd~ed.\hskip 1em plus
  %0.5em minus 0.4em\relax Harlow, England: Addison-Wesley, 1999.

\end{thebibliography}


% that's all folks
\end{document}


